\newpage
\chapter{METODE PENELITIAN} \label{Bab III}

\section{Alur Penelitian} \label{III.Alur}
Penelitian ini dilakukan melalui serangkaian tahapan eksperimen sistematis untuk mengevaluasi kinerja adaptasi arsitektur \textit{Pre-trained Audio Neural Networks} (PANNs). 
Pendekatan penelitian didasarkan pada metode \textit{Transfer Learning} dengan desain studi komparatif terhadap variasi representasi input audio. 
Secara rinci, kerangka kerja dan alur penelitian digambarkan pada diagram alur (\textit{flowchart}) dalam Gambar 3.1 berikut. \par

\begin{figure}[H] % Kalau menggunakan H, posisi gambar akan tepat dibawah teks
    \centering
    \includegraphics[width=1\textwidth]{figure/Flowchart_TA.jpg}
    \caption{Alur Penelitian}
    \label{fig:3.alur}
\end{figure}

\section{Penjabaran Langkah Penelitian} \label{III.Jabar Alur}
Berdasarkan diagram alur penelitian yang ditunjukkan pada gambar 3.1, penjelasan rinci terkait langkah-langkah penelitian yang dilakukan akan dijabarkan di bawah ini. \par

\subsection{Identifikasi Permasalahan} \label{III.Langkah 1}
Tahap awal penelitian dimulai dengan mengidentifikasi urgensi pengembangan sistem alat bantu bagi penyandang tunarungu yang berorientasi pada standar keselamatan tinggi. Alat bantu ini diharapkan tidak hanya mengamplifikasi sinyal, tetapi juga mampu mengenali suara bahaya di lingkungan sekitar secara otomatis.
Tantangan teknis utama yang dihadapi adalah keterbatasan jumlah sampel pada dataset yang spesifik untuk suara bahaya di lingkungan. Melatih model \textit{Deep Learning} dari awal (\textit{from scratch}) dengan data terbatas berisiko tinggi menyebabkan \textit{overfitting} dan performa yang buruk. Oleh karena itu, penggunaan arsitektur \textit{Pre-trained Audio Neural Networks} (PANNs) yang telah dilatih pada dataset raksasa AudioSet menjadi solusi strategis (\textit{Transfer Learning}) untuk mengatasi kelangkaan data tersebut.
Akan tetapi, penerapan strategi ini memunculkan permasalahan baru, yaitu belum diketahuinya representasi input audio manakah antara \textit{Raw Waveform} 1D, \textit{Spectrogram} 2D, atau \textit{Hybrid} yang paling optimal untuk mengadaptasi model PANNs tersebut. Ketidaktepatan pemilihan representasi input pada proses transfer pengetahuan ini berisiko menghambat performa model dalam mengenali bahaya secara akurat. \par

\subsection{Studi Literatur} \label{III.Langkah 2}
Tahap ini dilakukan untuk membangun landasan teoretis yang kokoh sebelum pra-pemrosesan data dan adaptasi model dimulai. Studi literatur difokuskan pada analisis mendalam terhadap arsitektur \textit{Pre-trained Audio Neural Networks} (PANNs), dengan perwakilan tiga model yang masing-masing memiliki representasi input yang berbeda, serta mekanisme \textit{Transfer Learning} untuk adaptasi domain. 
Selain itu, kajian juga dipusatkan pada karakteristik teknis dataset \textit{UrbanSound8K}, terutama terkait struktur metadata dan potensi kebocoran data (\textit{data leakage}). Pemahaman terhadap aspek-aspek tersebut menjadi acuan krusial dalam merancang strategi pembagian data (\textit{splitting strategy}) yang valid dan skenario komparasi representasi input yang objektif. 
Kajian literatur juga mencakup pemilihan metode evaluasi yang relevan untuk kasus klasifikasi multikelas pada dataset tidak seimbang, guna memastikan hasil eksperimen dapat diukur validitasnya secara objektif. \par

\subsection{Pengumpulan dan Pra-pemrosesan Data} \label{III.Langkah 3}
Data penelitian bersumber dari dataset publik \textit{UrbanSound8K}. Tahap pra-pemrosesan dilakukan melalui tiga langkah strategis untuk menjamin integritas data:

\begin{enumerate}
    \item \textbf{Penyaringan Kelas (\textit{Class Filtering}):} 
    Dari 10 kelas yang tersedia, dilakukan seleksi data untuk mengambil 4 kelas prioritas yang relevan dengan konteks keselamatan tunarungu, yaitu: \textit{car\_horn} (klakson), \textit{siren} (sirine), \textit{gun\_shot} (tembakan), dan \textit{dog\_bark} (gonggongan).

    \item \textbf{Strategi Partisi Anti-Bocor (\textit{Leakage-Free Splitting}):} 
    Dataset asli disusun berdasarkan metadata \texttt{fsID} (sumber rekaman) ke dalam 10 \textit{fold}. Untuk mencegah kebocoran data—dimana potongan audio dari sumber rekaman yang sama masuk ke data latih dan data uji sekaligus—penelitian ini menerapkan strategi pemetaan \textit{fold} (\textit{Fold Mapping}). Ke-10 \textit{fold} asli digabungkan dan dipetakan ulang menjadi 5 \textit{fold} eksperimen (Fold 1-2 menjadi Fold Baru 1, dst.) guna menjaga independensi data uji.

    \item \textbf{Penyeimbangan Kelas (\textit{Class Balancing}):} 
    Mengingat distribusi data mentah yang tidak seimbang (\textit{imbalanced}), diterapkan teknik \textit{Random Under-sampling} pada setiap \textit{fold} baru. Jumlah sampel pada kelas mayoritas dipangkas secara acak hingga setara dengan jumlah sampel pada kelas minoritas. Langkah ini krusial untuk mencegah model mengalami bias prediksi ke arah kelas yang dominan.
\end{enumerate}

Hasil akhir dari proses ini disimpan dalam format konfigurasi statis berupa \texttt{split.json} yang menjadi acuan baku bagi seluruh skenario pengujian model. \par

\subsection{Adaptasi Model} \label{III.Langkah 4}
Penelitian ini menggunakan tiga arsitektur dasar terbaik dari setiap representasi input nya, yaitu \textit{Res1dNet31}, \textit{ResNet38}, dan \textit{Wavegram-Logmel-CNN} dari pustaka PANNs yang telah dilatih sebelumnya (\textit{pre-trained}) pada dataset AudioSet. Untuk mengadaptasi model tersebut ke dalam domain permasalahan yang baru, dilakukan serangkaian langkah modifikasi pada arsitektur yang akan digunakan. Untuk langkah adaptasi yang lebih rinci, hal tersebut dijelaskan pada bagian 3.4.\par

\subsection{Evaluasi Model} \label{III.Langkah 5}
Setelah model diadaptasi, maka dilakukan evaluasi performa model menggunakan pendekatan \textit{Multi-metric Evaluation}.
Evaluasi kinerja model tidak hanya didasarkan pada metrik Akurasi (\textit{Accuracy}) semata, tetapi juga dianalisis secara mendalam menggunakan \textit{Confusion Matrix}. Penggunaan \textit{Confusion Matrix} sangat krusial dalam konteks sistem keselamatan untuk memetakan distribusi kesalahan prediksi—misalnya, seberapa sering suara sirine salah diklasifikasikan sebagai suara latar belakang atau kelas lainnya.
Hasil evaluasi dari kelima \textit{fold} kemudian dirata-rata untuk mendapatkan skor performa final. Berdasarkan data tersebut, dilakukan analisis komparatif untuk menyimpulkan representasi input manakah yang paling andal dan konsisten untuk diimplementasikan pada sistem alat bantu tunarungu. \par

\subsection{Hasil dan Kesimpulan} \label{III.Langkah 6}
Tahap penutup penelitian difokuskan pada interpretasi mendalam terhadap data kuantitatif yang diperoleh dari tahap evaluasi. Analisis dilakukan secara komparatif untuk menelaah kekuatan dan kelemahan dari masing-masing representasi input (\textit{Raw Waveform} vs \textit{Spectrogram} vs \textit{Hybrid}) dalam mendeteksi pola suara bahaya.

Berdasarkan analisis tersebut, ditarik kesimpulan akhir mengenai model mana yang memiliki performa paling optimal. Kesimpulan ini tidak hanya menjawab rumusan masalah secara statistik, tetapi juga memberikan rekomendasi teknis mengenai kelayakan implementasi model terpilih pada sistem alat bantu keselamatan bagi penyandang Tuna Rungu. \par

\section{Alat dan Bahan Tugas Akhir} \label{III.Alat dan Bahan}
Penelitian ini membutuhkan dukungan alat untuk proses adaptasi, pelatihan model, hingga evaluasi. Selain itu, ketersediaan bahan berupa data dan model dasar juga menjadi komponen krusial.

\subsection{Alat} \label{III.Alat}
Alat yang digunakan dalam penelitian ini adalah:

\begin{enumerate}
    \item \textit{Visual Studio Code} sebagai \textit{text editor}.
    \item Kaggle Notebook dengan \textit{GPU Runtime} sebagai lingkungan pengembangan dan eksekusi kode.
    \item Python versi 3.10.16
\end{enumerate}

\subsection{Bahan} \label{III.Bahan}
Bahan penelitian merujuk pada objek data digital yang diolah dan digunakan dalam eksperimen, meliputi:
\begin{enumerate}
    \item Dataset UrbanSound8K yang merupakan kumpulan data audio lingkungan yang terdiri dari 8732 potongan suara (\textit{slices}) yang terbagi dalam 10 kelas, beserta file metadata (\texttt{.csv}) yang memuat informasi \textit{fold} dan \textit{class ID}.
    \item \textit{Pre-trained Weights}, yaitu file bobot model PANNs, yaitu \textit{Res1dNet31.pth}, \textit{ResNet38.pth}, dan \textit{Wavegram\_Logmel\_CNN.pth}, yang telah dilatih sebelumnya pada dataset AudioSet. File ini digunakan sebagai inisialisasi awal dalam proses \textit{Transfer Learning}.
    \item Dokumen Referensi Utama yang berupa paper ilmiah "PANNs: Large-Scale Pretrained Audio Neural Networks for Audio Pattern Recognition" yang digunakan sebagai acuan standar arsitektur model \cite{kong2020panns}.
\end{enumerate}

\section{Metode Pengembangan} \label{III.Metode Pengembangan}
Bagian ini menjelaskan pendekatan teknis yang digunakan untuk membangun sistem klasifikasi suara. Berbeda dengan pengembangan perangkat lunak konvensional, penelitian ini berfokus pada pengembangan model cerdas berbasis data.

\subsection{Model Pengembangan}
Penelitian ini mengadopsi model pengembangan \textit{Machine Learning Pipeline}. Model ini dipilih karena penelitian bersifat eksperimental dan berorientasi pada data (\textit{data-driven}). Alur pengembangan tidak bersifat linier kaku, melainkan bersifat iteratif terutama pada tahap pelatihan dan evaluasi model untuk mencapai performa terbaik.

\subsection{Prosedur Pengembangan}
Secara spesifik, prosedur pengembangan model dalam penelitian ini terdiri dari empat tahapan utama:

\begin{enumerate}
    \item \textbf{Pra-pemrosesan Data (\textit{Data Preprocessing}):}
    Tahap ini mengubah data mentah menjadi format yang siap dilatih. Termasuk di dalamnya adalah konversi audio ke representasi visual (\textit{Spectrogram}) untuk input model 2D dan \textit{Hybrid}, serta pembagian data menggunakan skema \textit{5-Fold Cross Validation}.

    \item \textbf{Adaptasi Arsitektur (\textit{Model Adaptation}):}
    Tahap memodifikasi arsitektur PANNs agar sesuai dengan 4 kelas target. Lapisan klasifikasi asli (527 \textit{nodes}) diganti dengan lapisan \textit{Fully Connected} baru (4 \textit{nodes}).
    
    \item \textbf{Pelatihan Model (\textit{Model Training}):}
    Tahap pembelajaran mesin menggunakan strategi \textit{Transfer Learning}. Bobot model dilatih ulang (\textit{fine-tuning}) menggunakan optimasi \textit{AdamW} dan fungsi kerugian \textit{Cross-Entropy Loss}.

    \item \textbf{Evaluasi (\textit{Model Evaluation}):}
    Tahap pengujian kinerja model menggunakan data validasi yang belum pernah dilihat model sebelumnya untuk mengukur kemampuan generalisasi.
\end{enumerate}

\subsection{Cara Pengumpulan Data}
Sesuai dengan karakteristik penelitian \textit{Deep Learning} yang membutuhkan data dalam jumlah besar, metode pengumpulan data yang digunakan adalah Studi Dokumentasi Dataset Sekunder.

\begin{enumerate}
    \item \textbf{Sumber Data:} Data tidak dikumpulkan melalui perekaman langsung (primer), melainkan diunduh dari repositori publik \textit{UrbanSound8K}.
    \item \textbf{Teknik Sampling:} Pengambilan sampel dilakukan dengan teknik \textit{Purposive Sampling}, yaitu hanya mengambil data yang memenuhi kriteria inklusi: (a) Termasuk dalam 4 kelas bahaya (Klakson, Sirine, Tembakan, Gonggongan), dan (b) Memiliki durasi audio yang valid (< 4 detik).
\end{enumerate}

\subsection{Strategi \textit{Transfer Learning}}
Untuk mengatasi keterbatasan jumlah data pada dataset target, pengembangan model menerapkan teknik \textit{Partial Freezing}:
\begin{itemize}
    \item \textbf{Frozen Layers:} Lapisan awal (\textit{feature extractor}) dibekukan bobotnya agar fitur dasar suara yang sudah dipelajari dari AudioSet tidak hilang.
    \item \textbf{Trainable Layers:} Hanya lapisan akhir (\textit{classifier}) dan beberapa blok konvolusi terakhir yang diizinkan untuk memperbarui bobotnya selama proses pelatihan.
\end{itemize}

\section{Ilustrasi Perhitungan Metode} \label{III.Ilustrasi}
Pada bagian ini disajikan ilustrasi perhitungan metrik evaluasi berbasis \textit{Confusion Matrix} untuk kasus klasifikasi multikelas (4 kelas). Data yang digunakan bersifat hipotetik dengan jumlah sampel $N=20$, bertujuan untuk mengilustrasikan mekanisme perhitungan \textit{Accuracy}, \textit{Precision}, dan \textit{Recall}.

\subsection{Matriks Kebingungan (Confusion Matrix)}
Contoh hasil prediksi model terhadap 20 data uji yang terdiri dari kelas A (Klakson), B (Sirine), C (Tembakan), dan D (Gonggongan) disajikan pada Tabel 3.2.

\begin{table}[h!]
\centering
\caption{Contoh Matriks Kebingungan Multikelas (Data Hipotetik)}
\begin{tabular}{|c|c|c|c|c|c|}
\hline
\multicolumn{2}{|c|}{\multirow{2}{*}{}} & \multicolumn{4}{c|}{\textbf{Prediksi Model}} \\ \cline{3-6} 
\multicolumn{2}{|c|}{} & \textbf{A} & \textbf{B} & \textbf{C} & \textbf{D} \\ \hline
\multirow{4}{*}{\rotatebox{90}{\textbf{Aktual}}} 
& \textbf{A (Klakson)} & \textbf{4} & 1 & 0 & 0 \\ \cline{2-6} 
& \textbf{B (Sirine)} & 1 & \textbf{4} & 0 & 0 \\ \cline{2-6} 
& \textbf{C (Tembakan)} & 0 & 0 & \textbf{5} & 0 \\ \cline{2-6} 
& \textbf{D (Gonggongan)} & 0 & 1 & 0 & \textbf{4} \\ \hline
\end{tabular}
\end{table}

Keterangan diagonal utama (cetak tebal) menunjukkan prediksi yang benar (\textit{True Positive} untuk masing-masing kelas).

\subsection{Perhitungan Metrik}
Berdasarkan Tabel 3.2, total prediksi benar adalah jumlah diagonal utama: $4 + 4 + 5 + 4 = 17$.

\textbf{1. Akurasi Global (\textit{Overall Accuracy})} \\
Mengukur rasio prediksi benar terhadap total sampel.
\begin{equation}
    \text{Accuracy} = \frac{\sum \text{Diagonal}}{N} = \frac{17}{20} = 0,85 = 85\%
\end{equation}

\textbf{2. Perhitungan Per-Kelas (Contoh: Kelas B / Sirine)} \\
Dalam klasifikasi multikelas, \textit{Precision} dan \textit{Recall} dihitung secara terpisah untuk setiap kelas (\textit{One-vs-Rest}). Berikut contoh perhitungan untuk \textbf{Kelas B (Sirine)}:

\begin{itemize}
    \item \textbf{True Positive (TP):} Aktual Sirine diprediksi Sirine = 4.
    \item \textbf{False Positive (FP):} Aktual BUKAN Sirine, tapi diprediksi Sirine (Lihat Kolom B selain diagonal: 1 dari A + 1 dari D) = 2.
    \item \textbf{False Negative (FN):} Aktual Sirine, tapi diprediksi BUKAN Sirine (Lihat Baris B selain diagonal: 1 ke A) = 1.
\end{itemize}

Maka metrik untuk Kelas Sirine adalah:

\begin{equation}
    \text{Precision}_B = \frac{TP}{TP + FP} = \frac{4}{4 + 2} = \frac{4}{6} \approx 66,67\%
\end{equation}

\begin{equation}
    \text{Recall}_B = \frac{TP}{TP + FN} = \frac{4}{4 + 1} = \frac{4}{5} = 80\%
\end{equation}

Nilai akhir sistem (\textit{Macro Average}) diperoleh dengan merata-ratakan nilai metrik dari keempat kelas tersebut.

\section{Rancangan Pengujian} \label{III.Rancang_Uji}
Rancangan pengujian disusun untuk memverifikasi apakah sistem yang dikembangkan memenuhi spesifikasi kebutuhan, baik dari segi kemampuan deteksi (fungsional) maupun efisiensi komputasi (non-fungsional). Pengujian dilakukan secara komparatif terhadap ketiga variasi model yang diajukan.

\subsection{Lingkungan Pengujian}
Seluruh skenario pengujian dijalankan pada lingkungan yang seragam untuk menjamin validitas perbandingan (\textit{Apple-to-Apple Comparison}). Spesifikasi lingkungan pengujian adalah sebagai berikut:
\begin{enumerate}
    \item \textbf{Perangkat Keras:} Pengujian inferensi dilakukan menggunakan \textit{GPU Runtime} NVIDIA T4 pada platform Kaggle.
    \item \textbf{Perangkat Lunak:} Kerangka kerja \textit{PyTorch} dengan versi pustaka yang dikunci (\textit{fixed version}) untuk menghindari perbedaan optimasi pustaka.
\end{enumerate}

\subsection{Pengujian Fungsional (Kinerja Klasifikasi)}
Pengujian fungsional bertujuan untuk mengukur kehandalan model dalam menjalankan fungsi utamanya, yaitu mengklasifikasikan audio input ke dalam 4 kelas target secara akurat.
\begin{enumerate}
    \item \textbf{Metode:} Menggunakan \textit{Stratified 5-Fold Cross-Validation} untuk mendapatkan nilai performa yang objektif dan tidak bias oleh pembagian data tertentu.
    \item \textbf{Parameter Ukur:}
    \begin{itemize}
        \item \textbf{Macro F1-Score:} Sebagai metrik utama untuk melihat kinerja rata-rata yang adil pada dataset yang tidak seimbang.
        \item \textbf{Recall (Sensitivitas):} Khusus dipantau untuk kelas bahaya (Sirine dan Tembakan) guna memastikan aspek keselamatan pengguna.
        \item \textbf{Confusion Matrix:} Untuk memvalidasi apakah model mampu membedakan antar kelas yang mirip dengan benar.
    \end{itemize}
    \item \textbf{Kriteria Keberhasilan:} Model dinyatakan lolos uji fungsional jika mampu menghasilkan nilai \textit{F1-Score} di atas garis dasar (\textit{baseline}) model klasifikasi sederhana atau penelitian terdahulu yang relevan.
\end{enumerate}

\subsection{Pengujian Non-Fungsional (Efisiensi Sistem)}
Mengingat tujuan akhir sistem adalah sebagai alat bantu dengar yang bekerja secara \textit{real-time}, aspek performa non-fungsional menjadi krusial.
\begin{enumerate}
    \item \textbf{Waktu Inferensi (\textit{Inference Time}):}
    Mengukur rata-rata waktu yang dibutuhkan model untuk memproses satu sampel audio berdurasi 4 detik, mulai dari data masuk hingga keluar prediksi kelas. Pengukuran dilakukan dalam satuan milidetik (ms). Semakin kecil waktu inferensi, semakin layak model diterapkan pada sistem \textit{real-time}.
    
    \item \textbf{Ukuran Model (\textit{Model Size}):}
    Mengukur besaran file bobot model (\texttt{.pth}) yang dihasilkan setelah pelatihan. Parameter ini menguji efisiensi penyimpanan (\textit{storage}) jika nantinya model akan ditanamkan (\textit{embedded}) pada perangkat keras dengan memori terbatas.
\end{enumerate}
