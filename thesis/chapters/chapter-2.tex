\newpage
\chapter{TINJAUAN PUSTAKA} \label{Bab II}

\section{Tinjauan Pustaka} \label{II.Tinjauan}
Tinjauan pustaka dijelaskan dalam Tabel \ref{table:2.literasi} dan digunakan sebagai referensi berdasarkan penelitian yang berkaitan dengan penelitian ini.

\begin{longtable}{| b{0.05\textwidth}|p{0.2\textwidth}|p{0.2\textwidth}|p{0.15\textwidth}|p{0.25\textwidth}|} % Longtable berguna agar tabel dapat terpotong di halaman baru
	\caption{Literasi Penelitian Terdahulu}
	\label{table:2.literasi}\\
	\hline
	\textbf{No.} & \textbf{Judul} & \textbf{Masalah} & \textbf{Metode} & \textbf{Hasil} \\
	\hline
	\endfirsthead % Header tabel untuk halaman pertama
	\hline
	\textbf{No.} & \textbf{Judul} & \textbf{Masalah} & \textbf{Metode} & \textbf{Hasil} \\
	\hline
	\endhead % Header tabel untuk halaman selanjutnya (repeat header row)
	1. & Klasifikasi Suara Alat Musik Menggunakan CNN dan Mel-Spectrogram (2022) & Belum tersedia sistem klasifikasi otomatis untuk mengenali suara berbagai alat musik dari rekaman audio & CNN dengan input Mel-Spectrogram; preprocessing menggunakan Librosa, pelatihan model dilakukan dengan Keras & Sistem dapat mengklasifikasikan 6 jenis alat musik secara akurat dengan nilai akurasi tertinggi mencapai 94\%\\ 
	\hline
	2. & Deteksi Suara Chord Piano Menggunakan Metode Convolutional Neural Network (2022)	& Sulitnya mendeteksi akor piano secara otomatis dalam format suara	& CNN + Mel-spectrogram & Model CNN mampu mendeteksi akor mayor dan minor piano dengan akurasi >85\% \\ 
	\hline
	3. & Pengembangan Aplikasi Klasifikasi Suara Alat Musik Kalimba (2024) & Belum ada sistem klasifikasi otomatis suara alat musik kalimba & CNN + Mel-spectrogram & Aplikasi berhasil mengklasifikasi suara kalimba ke beberapa jenis nada dengan performa tinggi\\ 
	\hline
	4. & Recognising Bonang Barung Gamelan Instrument Playing Technique Using CNN (2025) & Kurangnya metode otomatis untuk mengenali teknik permainan bonang barung & CNN + Mel-spectrogram & Sistem mampu membedakan teknik pukulan bonang dengan akurasi yang cukup baik\\
	\hline
	5. & Klasifikasi Suara Instrumen Musik Tiup Menggunakan Metode CNN (2024) & Sulitnya mengidentifikasi jenis instrumen musik tiup dari file audio & CNN + Mel-spectrogram & Model berhasil mengklasifikasi berbagai instrumen tiup (saxophone, flute, trumpet) secara akurat\\
	\hline
\end{longtable}

\section{Dasar Teori} \label{II.Teori}

\subsection{Spectrogram} \label{II.Teori1}
Spectrogram merupakan representasi visual dari frekuensi sinyal audio terhadap waktu.

\subsection{Convolutional Neural Network (CNN)} \label{II.Teori2}
Convolutional Neural Network atau CNN adalah arsitektur jaringan saraf dalam pembelajaran mendalam, digunakan untuk klasifikasi citra dan spectrogram.

\subsection{Feature Extraction Audio} \label{II.Teori3}
Feature Extraction Audio merupakan proses mengambil fitur penting seperti mel-frequency cepstral coefficients (MFCC), chroma, atau spectrogram.

\subsection{Korean Ballad} \label{II.Teori4}
Korean Ballad adalah genre musik asal Korea yang dikenal lembut, emosional, dan memiliki aransemen instrumental kuat.