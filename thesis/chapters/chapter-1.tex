\newpage
\chapter{PENDAHULUAN} \label{Bab I}

\section{Latar Belakang} \label{I.Latar Belakang}
Ceritakan tentang penyandang tunanetra yang mengandalkan pendengaran untuk melakukan navigasi. \par

Kenalkan Environmental Sound Classification (ESC) menggunakan Deep Learning.\par

Disinilah Anda masuk. Sebutkan bahwa ada banyak metode: Input Suara Mentah (Waveform) dan Input Gambar (Spectrogram).
Banyak penelitian mengejar model yang Kompleks (Hybrid) agar canggih.
Masalahnya: Apakah model kompleks selalu cocok untuk dataset terbatas (seperti kasus spesifik ini)? Atau justru model yang lebih efisien (Spectrogram) lebih baik?\par

Inilah alasan kenapa Anda meneliti perbandingan 3 model ini. \par

\section{Rumusan Masalah} \label{I.Rumusan Masalah}
Berdasarkan latar belakang yang telah diuraikan, maka dirumuskan masalah dalam penelitian ini : \par

\begin{enumerate}[noitemsep]
	\item Bagaimana merancang bangun sistem klasifikasi suara bahaya (Siren, Car Horn, Dog Bark, Gun Shot) menggunakan arsitektur Pre-trained Audio Neural Networks (PANNs)? 
	\item Manakah arsitektur model yang memberikan performa terbaik (Akurasi, Presisi, Recall) di antara pendekatan Waveform (1D), Spectrogram (2D), dan Hybrid pada dataset UrbanSound8K?
	\item Bagaimana pengaruh kompleksitas model terhadap efektivitas klasifikasi pada jumlah data yang terbatas?
\end{enumerate}

\section{Tujuan Penelitian} \label{I.Tujuan}
Tujuan dari penelitian ini adalah : \par

\begin{enumerate}[noitemsep]
	\item Mengonversi audio musik instrumental Korean Ballad menjadi spectrogram.
	\item Membangun dan melatih model CNN untuk klasifikasi instrumen melodis.
	\item Menguji performa model CNN terhadap data uji yang representatif.
\end{enumerate}

\section{Batasan Masalah} \label{I.Batasan}
Batasan masalah yang didefinisikan dalam penelitian ini adalah sebagai berikut : \par

\begin{enumerate}[noitemsep]
	\item Penelitian hanya terbatas pada instrumen melodis seperti piano, biola, dan flute.
	\item Genre musik yang digunakan hanya dari Korean Ballad versi instrumental.
	\item Dataset yang digunakan adalah dataset bebas lisensi atau dikumpulkan sendiri.
	\item Tidak mempertimbangkan instrumen ritmis (drum, bass, dll).
\end{enumerate}

\section{Manfaat Penelitian} \label{I.Manfaat}
Manfaat dari penelitian ini adalah : \par

\begin{enumerate}[noitemsep]
	\item Memperluas pemahaman dalam bidang pemrosesan audio dan deep learning.
	\item Menambah ragam penelitian pada pemrosesan sinyal digital dan CNN.
	\item Menciptakan potensi implementasi pada aplikasi edukasi musik, pemrosesan audio otomatis, dan sistem klasifikasi musik cerdas.
\end{enumerate}


\section{Sistematika Penulisan} \label{I.Sistematika}
Sistematika penulisan berisi pembahasan apa yang akan ditulis disetiap Bab. Sistematika pada umumnya berupa paragraf yang setiap paragraf mencerminkan bahasan setiap Bab. \par

\subsection*{Bab I}
Bab ini berisikan penjelasan latar belakang dari topik penelitian yang berlangsung, rumusan masalah dari masalah yang dihadapi pada penjelasan di latar belakang, tujuan dari penelitian, batasan dari penelitian, manfaat dari hasil penelitian, dan sistematika penulisan tugas akhir. \par

\subsection*{Bab II}
Bab ini membahas mengenai tinjauan pustaka dari penelitan terdahulu dan dasar teori yang berkaitan dengan penelitian ini.

\subsection*{Bab III}
Bab ini berisikan penjelasan alur kerja sistem, alat dan data yang digunakan, metode yang digunakan, dan rancangan pengujian.

\subsection*{Bab IV}
Bab ini membahas hasil implementasi dan pengujian dari penelitian yang dilakukan, serta analisis dan evaluasi yang dapat dipetik dari hasil.

\subsection*{Bab V}
Bab ini membahas kesimpulan dari hasil penelitian dan juga saran untuk penelitian selanjutnya.