\newpage
\chapter{PENDAHULUAN} \label{Bab I}

\section{Latar Belakang} \label{I.Latar Belakang}
[PARAGRAF 1 : Masalah Nyata] Dalam konteks mobilitas di daerah perkotaan, indra pendengaran merupakan mekanisme deteksi alami yang cukup penting untuk mengetahui kondisi lingkungan di sekitarnya. 
Namun, fungsi indra tersebut tidak dimiliki oleh penyandang Tuna Rungu, yang hanya bisa mengandalkan penglihatan mereka untuk memahami situasi di sekitar. 
Ketergantungan penuh pada aspek visual ini dapat menjadi kerentanan serius, mengingat mereka memiliki keterbatasan sudut pandang dan dan tidak dapat memantau kondisi di luar jangkauan penglihatan. 
Akibatnya, ancaman yang muncul dari titik buta, seperti gonggongan anjing yang mengejar atau klakson kendaraan yang melaju kencang dari arah belakang, seringkali terlambat disadari, sehingga berpotensi menimbulkan kecelakaan fatal. \par

[PARAGRAF 2 : Solusi Lama] Saat ini, penyandang Tuna Rungu umumnya mengandalkan Alat Bantu Dengar (ABD) sebagai instrumen utama untuk menunjang komunikasi verbal. 
Meskipun efektif untuk komunikasi verbal jarak dekat, alat ini memiliki keterbatasan signifikan dalam konteks keselamatan di luar ruangan.  
Hal ini disebabkan oleh penurunan selektivitas frekuensi, yang membuat penyandang Tuna Rungu kesulitan memisahkan suara utama dari suara lain yang saling bertabrakan (Cari sitasi tentang ini).
ABD hanya berfungsi untuk amplifikasi seluruh sinyal suara yang masuk, sehingga kebisingan latar belakang pun tak terelakkan ikut teramplifikasi. 
Akibatnya, sinyal ancaman penting seringkali tertutup oleh suara-suara lainnya, yang berdampak pada hilangnya kewaspadaan situasional pengguna. \par

[PARAGRAF 3 : Solusi Ditawarkan] Untuk mengatasi kendala tersebut, teknologi klasifikasi suara lingkungan (Environmental Sound Classification/ESC) berbasis deep learning hadir sebagai solusi bagi para penyandang Tuna Rungu dalam mengenali suara-suara bahaya di sekitar mereka. 
Dengan memanfaatkan model deep learning yang telah dilatih untuk mengenali berbagai jenis suara lingkungan, sistem ESC dapat mendeteksi dan mengklasifikasikan suara-suara bahaya seperti sirene ambulans, klakson mobil, gonggongan anjing, atau suara tembakan. 
Setelah suara terdeteksi dan diklasifikasikan, sistem dapat memberikan notifikasi dalam bentuk visual atau getaran kepada pengguna Tuna Rungu, sehingga mereka dapat segera menyadari ancaman dan mengambil tindakan antisipasi. 
Dengan demikian, teknologi ESC berbasis deep learning memiliki potensi besar untuk meningkatkan keselamatan dan kualitas hidup penyandang Tuna Rungu dalam mobilitas sehari-hari di lingkungan perkotaan.\par

[PARAGRAF 4 : PANNs sebagai Solusi] Untuk membangun model ESC, diperlukan latihan berulang dengan banyaknya dataset yang ada. 
Maka dari itu, PANNs hadir sebagai solusi yang menyediakan model pre-trained yang dapat digunakan sebagai dasar pelatihan model ESC. 
Dengan menggunakan PANNs, proses pelatihan dapat dilakukan secara lebih efisien dan efektif karena model ini telah melalui proses pelatihan yang intensif pada kumpulan dataset. \par

[PARAGRAF 5 : Inti Skripsi] Dalam merancang ESC yang efektif dan andal, banyak jenis metode deep learning yang digunakan untuk membedakan suara-suara tersebut. 
Hal ini menjadi sangat penting, dikarenakan metode ESC yang buruk juga dapat berakibat fatal bagi keselamatan para pengguna. 
Maka dari itu, penelitian ini bertujuan untuk membandingkan performa dari tiga pendekatan berbeda dalam klasifikasi suara lingkungan: Input Suara Mentah (Waveform), Input Gambar (Spectrogram), dan Pendekatan Hybrid. \par

\section{Rumusan Masalah} \label{I.Rumusan Masalah}
Berdasarkan latar belakang yang telah diuraikan, maka dirumuskan masalah dalam penelitian ini : \par

\begin{enumerate}[noitemsep]
	\item Bagaimana merancang bangun sistem klasifikasi suara bahaya (Siren, Car Horn, Dog Bark, Gun Shot) menggunakan arsitektur Pre-trained Audio Neural Networks (PANNs)? 
	\item Manakah arsitektur model yang memberikan performa terbaik (Akurasi, Presisi, Recall) di antara pendekatan Waveform (1D), Spectrogram (2D), dan Hybrid pada dataset UrbanSound8K?
	\item Bagaimana pengaruh kompleksitas model terhadap efektivitas klasifikasi pada jumlah data yang terbatas?
\end{enumerate}

\section{Tujuan Penelitian} \label{I.Tujuan}
Tujuan dari penelitian ini adalah : \par

\begin{enumerate}[noitemsep]
	\item Mengonversi audio musik instrumental Korean Ballad menjadi spectrogram.
	\item Membangun dan melatih model CNN untuk klasifikasi instrumen melodis.
	\item Menguji performa model CNN terhadap data uji yang representatif.
\end{enumerate}

\section{Batasan Masalah} \label{I.Batasan}
Batasan masalah yang didefinisikan dalam penelitian ini adalah sebagai berikut : \par

\begin{enumerate}[noitemsep]
	\item Penelitian hanya terbatas pada instrumen melodis seperti piano, biola, dan flute.
	\item Genre musik yang digunakan hanya dari Korean Ballad versi instrumental.
	\item Dataset yang digunakan adalah dataset bebas lisensi atau dikumpulkan sendiri.
	\item Tidak mempertimbangkan instrumen ritmis (drum, bass, dll).
\end{enumerate}

\section{Manfaat Penelitian} \label{I.Manfaat}
Manfaat dari penelitian ini adalah : \par

\begin{enumerate}[noitemsep]
	\item Memperluas pemahaman dalam bidang pemrosesan audio dan deep learning.
	\item Menambah ragam penelitian pada pemrosesan sinyal digital dan CNN.
	\item Menciptakan potensi implementasi pada aplikasi edukasi musik, pemrosesan audio otomatis, dan sistem klasifikasi musik cerdas.
\end{enumerate}


\section{Sistematika Penulisan} \label{I.Sistematika}
Sistematika penulisan berisi pembahasan apa yang akan ditulis disetiap Bab. Sistematika pada umumnya berupa paragraf yang setiap paragraf mencerminkan bahasan setiap Bab. \par

\subsection*{Bab I}
Bab ini berisikan penjelasan latar belakang dari topik penelitian yang berlangsung, rumusan masalah dari masalah yang dihadapi pada penjelasan di latar belakang, tujuan dari penelitian, batasan dari penelitian, manfaat dari hasil penelitian, dan sistematika penulisan tugas akhir. \par

\subsection*{Bab II}
Bab ini membahas mengenai tinjauan pustaka dari penelitan terdahulu dan dasar teori yang berkaitan dengan penelitian ini.

\subsection*{Bab III}
Bab ini berisikan penjelasan alur kerja sistem, alat dan data yang digunakan, metode yang digunakan, dan rancangan pengujian.

\subsection*{Bab IV}
Bab ini membahas hasil implementasi dan pengujian dari penelitian yang dilakukan, serta analisis dan evaluasi yang dapat dipetik dari hasil.

\subsection*{Bab V}
Bab ini membahas kesimpulan dari hasil penelitian dan juga saran untuk penelitian selanjutnya.