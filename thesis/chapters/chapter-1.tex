\newpage
\chapter{PENDAHULUAN} \label{Bab I}

\section{Latar Belakang} \label{I.Latar Belakang}
Dalam konteks mobilitas di daerah perkotaan, indra pendengaran merupakan mekanisme deteksi alami yang cukup penting untuk mengetahui kondisi lingkungan di sekitarnya. 
Namun, fungsi indra tersebut tidak dimiliki oleh penyandang Tuna Rungu, yang hanya bisa mengandalkan penglihatan mereka untuk memahami situasi di sekitar. 
Ketergantungan penuh pada aspek visual ini dapat menjadi kerentanan serius, mengingat mereka memiliki keterbatasan sudut pandang dan tidak dapat memantau kondisi di luar jangkauan penglihatan. 
Akibatnya, ancaman yang muncul dari titik buta, seperti gonggongan anjing yang mengejar atau klakson kendaraan yang melaju kencang dari arah belakang, seringkali terlambat disadari, sehingga berpotensi menimbulkan kecelakaan fatal. \par

Saat ini, penyandang Tuna Rungu umumnya mengandalkan Alat Bantu Dengar (ABD) sebagai instrumen utama untuk menunjang komunikasi verbal. 
Meskipun efektif untuk komunikasi verbal jarak dekat, alat ini memiliki keterbatasan signifikan dalam konteks keselamatan di luar ruangan.  
Hal ini disebabkan oleh penurunan selektivitas frekuensi, yang membuat penyandang Tuna Rungu kesulitan memisahkan suara utama dari suara lain yang saling bertabrakan (Cari sitasi tentang ini).
ABD hanya berfungsi untuk amplifikasi seluruh sinyal suara yang masuk, sehingga kebisingan latar belakang pun tak terelakkan ikut teramplifikasi. 
Akibatnya, sinyal ancaman penting seringkali tertutup oleh suara-suara lainnya, yang berdampak pada hilangnya kewaspadaan situasional pengguna. \par

Untuk mengatasi kendala tersebut, teknologi klasifikasi suara lingkungan (Environmental Sound Classification/ESC) berbasis deep learning hadir sebagai solusi bagi para penyandang Tuna Rungu dalam mengenali suara-suara bahaya di sekitar mereka. 
Dengan memanfaatkan model deep learning yang telah dilatih untuk mengenali berbagai jenis suara lingkungan, sistem ESC dapat mendeteksi dan mengklasifikasikan suara-suara bahaya seperti sirene ambulans, klakson mobil, gonggongan anjing, atau suara tembakan. 
Setelah suara terdeteksi dan diklasifikasikan, sistem dapat memberikan notifikasi dalam bentuk visual atau getaran kepada pengguna Tuna Rungu, sehingga mereka dapat segera menyadari ancaman dan mengambil tindakan antisipasi. 
Dengan demikian, teknologi ESC berbasis deep learning memiliki potensi besar untuk meningkatkan keselamatan dan kualitas hidup penyandang Tuna Rungu dalam mobilitas sehari-hari di lingkungan perkotaan.\par

Namun, pengembangan model ESC yang mampu mendeteksi bahaya secara akurat menghadapi tantangan teknis, terutama terkait kebutuhan dataset berskala besar. 
Oleh karena itu, metode transfer learning menjadi solusi efektif untuk melatih model deteksi, dengan cara mengadaptasi arsitektur Pre-trained Audio Neural Networks (PANNs) yang telah memiliki pengetahuan dari dataset masif. 
Meskipun begitu, performa model ini sangat bergantung pada pemilihan representasi masukan suara, terutama terkait perbedaan efektivitas antara representasi Raw Waveform (satu dimensi) dan Spectrogram (dua dimensi). Selain itu, terdapat pula pendekatan Hybrid yang menggabungkan kedua representasi tersebut untuk memaksimalkan akurasi deteksi. \par

Guna memastikan tingkat akurasi deteksi yang maksimal bagi keselamatan pengguna, penelitian ini bertujuan untuk mengevaluasi dan membandingkan kinerja dari tiga pendekatan representasi input, yaitu: Raw Waveform, Spectrogram, dan Hybrid.
Melalui perbandingan ini, diharapkan dapat ditemukan model terbaik yang tidak hanya memiliki akurasi tinggi, tetapi juga mampu meminimalisir kesalahan deteksi yang fatal. 
Hasil akhir dari penelitian ini diharapkan dapat memberikan kontribusi nyata dalam pengembangan teknologi asistif berbasis AI yang andal, guna meningkatkan kewaspadaan dan jaminan keselamatan bagi komunitas Tuna Rungu.\par

\section{Rumusan Masalah} \label{I.Rumusan Masalah}
Berdasarkan latar belakang yang telah diuraikan, maka dirumuskan masalah dalam penelitian ini : \par

\begin{enumerate}[noitemsep]
	\item Bagaimana perbandingan performa model \textit{Pre-trained Audio Neural Networks (PANNs)} dalam mengklasifikasikan suara bahaya ketika dilatih menggunakan representasi input \textit{Raw Waveform}, \textit{Spectrogram}, dan \textit{Hybrid}?
	\item Manakah pendekatan representasi input yang memberikan performa terbaik (ditinjau dari \textit{Accuracy}, \textit{Precision}, \textit{Recall}) di antara ketiga pendekatan tersebut pada ketersediaan data yang terbatas?
\end{enumerate}

\section{Tujuan Penelitian} \label{I.Tujuan}
Tujuan dari penelitian ini adalah : \par

\begin{enumerate}[noitemsep]
	\item Membandingkan performa model \textit{Pre-trained Audio Neural Networks (PANNs)} dalam mengklasifikasikan suara bahaya ketika dilatih menggunakan representasi input \textit{Raw Waveform}, \textit{Spectrogram}, dan \textit{Hybrid}.
	\item Menentukan pendekatan representasi input yang memberikan performa terbaik (ditinjau dari \textit{Accuracy}, \textit{Precision}, \textit{Recall}) di antara ketiga pendekatan tersebut pada ketersediaan data yang terbatas.
\end{enumerate}

\section{Batasan Masalah} \label{I.Batasan}
Batasan masalah yang didefinisikan dalam penelitian ini adalah sebagai berikut : \par

\begin{enumerate}[noitemsep]
	\item Data yang digunakan bersumber dari dataset publik UrbanSound8K, dengan seleksi sampel yang diseimbangkan pada 4 kelas suara yang berpotensi membahayakan keselamatan, yaitu \textit{Siren} (Sirene), \textit{Car Horn} (Klakson Mobil), \textit{Dog Bark} (Gonggongan Anjing), dan \textit{Gun Shot} (Tembakan).
	\item Metode klasifikasi menggunakan teknik \textit{Transfer Learning} dengan arsitektur \textit{Pre-trained Audio Neural Networks (PANNs)}. Penelitian ini tidak membangun arsitektur dari nol, melainkan melakukan teknik \textit{fine-tuning} pada bobot parameter model PANNs yang telah dilatih sebelumnya.
	\item Penelitian ini berfokus pada analisis performa model dan komparasi representasi input (\textit{Raw Waveform}, \textit{Spectrogram}, \textit{Hybrid}). Implementasi antarmuka pengguna (\textit{User Interface}), aplikasi \textit{mobile}, atau sistem peringatan dini berbasis perangkat keras (\textit{hardware}) tidak termasuk dalam lingkup penelitian ini.
	\item Performa model dievaluasi berdasarkan parameter \textit{Accuracy}, \textit{Precision}, \textit{Recall}, \textit{F1-Score}, dan analisis \textit{Confusion Matrix}.
	\item Implementasi kode dan logika model dibangun menggunakan bahasa pemrograman \textit{Python} pada \textit{Visual Studio Code}. Proses pelatihan model (\textit{training}) dieksekusi menggunakan platform komputasi awan \textit{Kaggle} untuk memanfaatkan akselerasi GPU (\textit{Graphics Processing Unit}), dengan bantuan \textit{library} pengolahan audio dan \textit{Deep Learning} standar (\textit{PyTorch} dan \textit{Librosa}).
\end{enumerate}

\section{Manfaat Penelitian} \label{I.Manfaat}
Manfaat dari penelitian ini adalah : \par

\begin{enumerate}[noitemsep]
	\item Memberikan bukti empiris terkait efektivitas metode \textit{Transfer Learning} pada arsitektur PANNs serta perbandingan performa antara representasi input \textit{Raw Waveform}, \textit{Spectrogram}, dan \textit{Hybrid} dalam mengatasi keterbatasan dataset (\textit{limited dataset}).
	\item Berkontribusi dalam pengembangan teknologi asistif berbasis AI yang dapat meningkatkan keselamatan dan kemandirian mobilitas penyandang Tuna Rungu melalui deteksi suara bahaya yang akurat.
	\item Menjadi referensi bagi penelitian selanjutnya atau pengembang aplikasi dalam menentukan konfigurasi model yang paling optimal untuk diterapkan pada sistem peringatan dini.
\end{enumerate}


\section{Sistematika Penulisan} \label{I.Sistematika}
Sistematika penulisan berisi pembahasan apa yang akan ditulis disetiap Bab. Sistematika pada umumnya berupa paragraf yang setiap paragraf mencerminkan bahasan setiap Bab. \par

\subsection*{Bab I}
Bab ini berisikan penjelasan latar belakang dari topik penelitian yang berlangsung, rumusan masalah dari masalah yang dihadapi pada penjelasan di latar belakang, tujuan dari penelitian, batasan dari penelitian, manfaat dari hasil penelitian, dan sistematika penulisan tugas akhir. \par

\subsection*{Bab II}
Bab ini membahas mengenai tinjauan pustaka dari penelitan terdahulu dan dasar teori yang berkaitan dengan penelitian ini.

\subsection*{Bab III}
Bab ini berisikan penjelasan alur kerja sistem, alat dan data yang digunakan, metode yang digunakan, dan rancangan pengujian.

\subsection*{Bab IV}
Bab ini membahas hasil implementasi dan pengujian dari penelitian yang dilakukan, serta analisis dan evaluasi yang dapat dipetik dari hasil.

\subsection*{Bab V}
Bab ini membahas kesimpulan dari hasil penelitian dan juga saran untuk penelitian selanjutnya.