\newpage
\chapter{PENDAHULUAN} \label{Bab I}

\section{Latar Belakang} \label{I.Latar Belakang}
Dalam aktivitas mobilitas di daerah perkotaan, indra pendengaran merupakan mekanisme deteksi alami yang cukup penting untuk memantau kondisi lingkungan di sekitarnya. 
Namun, fungsi indra tersebut tidak dimiliki oleh penyandang Tuna Rungu yang hanya bisa mengandalkan penglihatan mereka untuk memahami situasi di sekitar. 
Ketergantungan penuh pada visual ini dapat menjadi kerentanan serius dikarenakan mereka memiliki keterbatasan sudut pandang dan tidak mengetahui apa situasi yang ada di luar penglihatan mereka. 
Akibatnya, seringkali ancaman yang muncul dari titik buta, seperti gonggongan anjing yang mengejar atau klakson kendaraan yang melaju kencang dari arah belakang, berpotensi menimbulkan kecelakaan fatal bagi mereka. \par

Saat ini, penyandang Tuna Rungu dibantu oleh alat asistif yaitu Alat Bantu Dengar (ABD). 
Meskipun efektif untuk komunikasi jarak dekat, alat ini memiliki keterbatasan dalam konteks keselamatan di luar ruangan. 
Tuna Rungu memiliki keterbatasan dalam membedakan suara yang terdengar oleh ABD akibat penurunan selektivitas frekuensi (Cari kutipan mengenai keterbatasan Tuna Rungu membedakan suara-suara). 
ABD hanya berfungsi untuk amplifikasi sinyal yang didengar, sehingga kebisingan latar belakang ikut teramplifiikasi. 
Akibatnya, suara yang menjadi ancaman penting pun tertutup oleh suara-suara lainnya sehingga hilangnya kewaspadaan situasional. \par

Disinilah peran teknologi klasifikasi suara lingkungan (Environmental Sound Classification/ESC) berbasis deep learning yang dapat menjadi solusi bagi para penyandang Tuna Rungu dalam mengenali suara-suara bahaya di sekitar mereka. 
Dengan memanfaatkan model deep learning yang telah dilatih untuk mengenali berbagai jenis suara lingkungan, sistem ESC dapat mendeteksi dan mengklasifikasikan suara-suara penting seperti sirene ambulans, klakson mobil, gonggongan anjing, atau suara tembakan. 
Setelah suara terdeteksi dan diklasifikasikan, sistem dapat memberikan notifikasi dalam bentuk visual atau getaran kepada pengguna Tuna Rungu, sehingga mereka dapat segera menyadari dan merespon potensi bahaya di sekitar mereka. 
Dengan demikian, teknologi ESC berbasis deep learning memiliki potensi besar untuk meningkatkan keselamatan dan kualitas hidup penyandang Tuna Rungu dalam mobilitas sehari-hari mereka di lingkungan perkotaan.\par

Dalam merancang ESC yang efektif dan andal, banyak jenis metode deep learning yang digunakan untuk membedakan suara-suara tersebut. Hal ini menjadi sangat penting, dikarenakan metode ESC yang buruk juga dapat berakibat fatal bagi keselamatan para pengguna. Maka dari itu, penelitian ini bertujuan untuk membandingkan performa dari tiga pendekatan berbeda dalam klasifikasi suara lingkungan: Input Suara Mentah (Waveform), Input Gambar (Spectrogram), dan Pendekatan Hybrid. \par

Inilah alasan kenapa Anda meneliti perbandingan 3 model ini. \par

\section{Rumusan Masalah} \label{I.Rumusan Masalah}
Berdasarkan latar belakang yang telah diuraikan, maka dirumuskan masalah dalam penelitian ini : \par

\begin{enumerate}[noitemsep]
	\item Bagaimana merancang bangun sistem klasifikasi suara bahaya (Siren, Car Horn, Dog Bark, Gun Shot) menggunakan arsitektur Pre-trained Audio Neural Networks (PANNs)? 
	\item Manakah arsitektur model yang memberikan performa terbaik (Akurasi, Presisi, Recall) di antara pendekatan Waveform (1D), Spectrogram (2D), dan Hybrid pada dataset UrbanSound8K?
	\item Bagaimana pengaruh kompleksitas model terhadap efektivitas klasifikasi pada jumlah data yang terbatas?
\end{enumerate}

\section{Tujuan Penelitian} \label{I.Tujuan}
Tujuan dari penelitian ini adalah : \par

\begin{enumerate}[noitemsep]
	\item Mengonversi audio musik instrumental Korean Ballad menjadi spectrogram.
	\item Membangun dan melatih model CNN untuk klasifikasi instrumen melodis.
	\item Menguji performa model CNN terhadap data uji yang representatif.
\end{enumerate}

\section{Batasan Masalah} \label{I.Batasan}
Batasan masalah yang didefinisikan dalam penelitian ini adalah sebagai berikut : \par

\begin{enumerate}[noitemsep]
	\item Penelitian hanya terbatas pada instrumen melodis seperti piano, biola, dan flute.
	\item Genre musik yang digunakan hanya dari Korean Ballad versi instrumental.
	\item Dataset yang digunakan adalah dataset bebas lisensi atau dikumpulkan sendiri.
	\item Tidak mempertimbangkan instrumen ritmis (drum, bass, dll).
\end{enumerate}

\section{Manfaat Penelitian} \label{I.Manfaat}
Manfaat dari penelitian ini adalah : \par

\begin{enumerate}[noitemsep]
	\item Memperluas pemahaman dalam bidang pemrosesan audio dan deep learning.
	\item Menambah ragam penelitian pada pemrosesan sinyal digital dan CNN.
	\item Menciptakan potensi implementasi pada aplikasi edukasi musik, pemrosesan audio otomatis, dan sistem klasifikasi musik cerdas.
\end{enumerate}


\section{Sistematika Penulisan} \label{I.Sistematika}
Sistematika penulisan berisi pembahasan apa yang akan ditulis disetiap Bab. Sistematika pada umumnya berupa paragraf yang setiap paragraf mencerminkan bahasan setiap Bab. \par

\subsection*{Bab I}
Bab ini berisikan penjelasan latar belakang dari topik penelitian yang berlangsung, rumusan masalah dari masalah yang dihadapi pada penjelasan di latar belakang, tujuan dari penelitian, batasan dari penelitian, manfaat dari hasil penelitian, dan sistematika penulisan tugas akhir. \par

\subsection*{Bab II}
Bab ini membahas mengenai tinjauan pustaka dari penelitan terdahulu dan dasar teori yang berkaitan dengan penelitian ini.

\subsection*{Bab III}
Bab ini berisikan penjelasan alur kerja sistem, alat dan data yang digunakan, metode yang digunakan, dan rancangan pengujian.

\subsection*{Bab IV}
Bab ini membahas hasil implementasi dan pengujian dari penelitian yang dilakukan, serta analisis dan evaluasi yang dapat dipetik dari hasil.

\subsection*{Bab V}
Bab ini membahas kesimpulan dari hasil penelitian dan juga saran untuk penelitian selanjutnya.