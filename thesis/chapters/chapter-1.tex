\newpage
\chapter{PENDAHULUAN} \label{Bab I}

\section{Latar Belakang} \label{I.Latar Belakang}
Perkembangan teknologi di bidang pemrosesan sinyal digital, khususnya audio, telah memungkinkan berbagai aplikasi seperti pengenalan suara, klasifikasi genre musik, hingga identifikasi instrumen musik. Salah satu aspek yang menarik untuk diteliti adalah klasifikasi instrumen musik berdasarkan karakteristik audio-nya. \par

Genre Korean Ballad merupakan salah satu genre musik populer dari Korea Selatan yang dikenal dengan alunan melodi yang emosional dan harmoni yang lembut. Dalam versi instrumental dari genre ini, instrumen melodis seperti piano, biola, dan flute berperan penting dalam menyampaikan suasana lagu tanpa kehadiran vokal.\par

Dalam tugas akhir ini, dilakukan penelitian untuk mengklasifikasi jenis instrumen melodis pada musik instrumental Korean Ballad. Teknologi yang digunakan adalah Convolutional Neural Network (CNN), yang mampu mempelajari fitur dari representasi spectrogram, yaitu visualisasi frekuensi terhadap waktu dari sinyal audio.\par

Dengan memanfaatkan CNN dan spectrogram, sistem ini diharapkan mampu membedakan instrumen-instrumen melodis berdasarkan pola-pola karakteristik frekuensinya, sehingga bisa digunakan sebagai dasar sistem klasifikasi otomatis di bidang musik. \par

\section{Rumusan Masalah} \label{I.Rumusan Masalah}
Berdasarkan latar belakang yang telah diuraikan, maka dirumuskan masalah dalam penelitian ini : \par

\begin{enumerate}[noitemsep]
	\item Bagaimana cara memproses audio instrumental Korean Ballad menjadi representasi spectrogram? 
	\item Bagaimana membangun dan melatih model CNN untuk mengklasifikasikan instrumen melodis dari spectrogram tersebut?
	\item Bagaimana menguji dan mengevaluasi akurasi klasifikasi instrumen melodis yang dihasilkan?
\end{enumerate}

\section{Tujuan Penelitian} \label{I.Tujuan}
Tujuan dari penelitian ini adalah : \par

\begin{enumerate}[noitemsep]
	\item Mengonversi audio musik instrumental Korean Ballad menjadi spectrogram.
	\item Membangun dan melatih model CNN untuk klasifikasi instrumen melodis.
	\item Menguji performa model CNN terhadap data uji yang representatif.
\end{enumerate}

\section{Batasan Masalah} \label{I.Batasan}
Batasan masalah yang didefinisikan dalam penelitian ini adalah sebagai berikut : \par

\begin{enumerate}[noitemsep]
	\item Penelitian hanya terbatas pada instrumen melodis seperti piano, biola, dan flute.
	\item Genre musik yang digunakan hanya dari Korean Ballad versi instrumental.
	\item Dataset yang digunakan adalah dataset bebas lisensi atau dikumpulkan sendiri.
	\item Tidak mempertimbangkan instrumen ritmis (drum, bass, dll).
\end{enumerate}

\section{Manfaat Penelitian} \label{I.Manfaat}
Manfaat dari penelitian ini adalah : \par

\begin{enumerate}[noitemsep]
	\item Memperluas pemahaman dalam bidang pemrosesan audio dan deep learning.
	\item Menambah ragam penelitian pada pemrosesan sinyal digital dan CNN.
	\item Menciptakan potensi implementasi pada aplikasi edukasi musik, pemrosesan audio otomatis, dan sistem klasifikasi musik cerdas.
\end{enumerate}


\section{Sistematika Penulisan} \label{I.Sistematika}
Sistematika penulisan berisi pembahasan apa yang akan ditulis disetiap Bab. Sistematika pada umumnya berupa paragraf yang setiap paragraf mencerminkan bahasan setiap Bab. \par

\subsection*{Bab I}
Bab ini berisikan penjelasan latar belakang dari topik penelitian yang berlangsung, rumusan masalah dari masalah yang dihadapi pada penjelasan di latar belakang, tujuan dari penelitian, batasan dari penelitian, manfaat dari hasil penelitian, dan sistematika penulisan tugas akhir. \par

\subsection*{Bab II}
Bab ini membahas mengenai tinjauan pustaka dari penelitan terdahulu dan dasar teori yang berkaitan dengan penelitian ini.

\subsection*{Bab III}
Bab ini berisikan penjelasan alur kerja sistem, alat dan data yang digunakan, metode yang digunakan, dan rancangan pengujian.

\subsection*{Bab IV}
Bab ini membahas hasil implementasi dan pengujian dari penelitian yang dilakukan, serta analisis dan evaluasi yang dapat dipetik dari hasil.

\subsection*{Bab V}
Bab ini membahas kesimpulan dari hasil penelitian dan juga saran untuk penelitian selanjutnya.