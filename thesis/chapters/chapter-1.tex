\newpage
\chapter{PENDAHULUAN} \label{Bab I}

\section{Latar Belakang} \label{I.Latar Belakang}

% PARAGRAF 1: URGENSI MASALAH (SOSIAL)
Dalam konteks mobilitas fisik di daerah perkotaan, indra pendengaran merupakan mekanisme deteksi alami yang cukup penting untuk mengetahui kondisi lingkungan di sekitar. 
Namun, fungsi indra tersebut tidak dimiliki oleh penyandang Tuna Rungu, yang hanya bisa mengandalkan penglihatan mereka untuk memantau keadaan. 
Ketergantungan penuh pada aspek visual ini dapat menjadi kerentanan serius, mengingat mereka memiliki keterbatasan sudut pandang dan tidak dapat memantau kondisi di luar jangkauan penglihatan. 
Akibatnya, ancaman yang muncul dari titik buta (\textit{blind spot}), seperti gonggongan anjing yang mengejar atau klakson kendaraan yang melaju kencang dari arah belakang, seringkali terlambat disadari, sehingga berpotensi menimbulkan kecelakaan fatal. \par

% PARAGRAF 2: KELEMAHAN SOLUSI EKSISTING (ABD)
Saat ini, Alat Bantu Dengar (ABD) merupakan perangkat yang umum digunakan untuk menunjang komunikasi verbal penyandang Tuna Rungu. 
Meskipun efektif untuk komunikasi verbal jarak dekat, alat ini memiliki keterbatasan signifikan dalam konteks keselamatan di luar ruangan. 
Hal ini disebabkan oleh penurunan selektivitas frekuensi (\textit{reduced frequency selectivity}) yang umum terjadi pada gangguan pendengaran sensorineural, sehingga menyulitkan pemisahan sinyal suara utama dari kebisingan latar belakang yang tumpang tindih \cite{moore1996perceptual}. 
Kondisi ini diperburuk oleh keterbatasan teknis ABD, di mana sekadar amplifikasi sinyal suara tidak cukup untuk mengembalikan kemampuan pemilahan suara secara alami. 
Akibatnya, sinyal ancaman penting seringkali tertutup oleh suara-suara lainnya, yang berdampak pada hilangnya kewaspadaan situasional pengguna. \par

% PARAGRAF 3: SOLUSI TEKNOLOGI (ESC)
Untuk mengatasi kendala tersebut, teknologi klasifikasi suara lingkungan (\textit{Environmental Sound Classification}/ESC) berbasis \textit{Deep Learning} hadir sebagai solusi bagi para penyandang Tuna Rungu dalam mengenali suara-suara bahaya di sekitar mereka. 
Dengan memanfaatkan model \textit{Deep Learning} yang telah dilatih untuk mengenali berbagai jenis suara lingkungan, sistem ESC dapat mendeteksi dan mengklasifikasikan suara-suara bahaya seperti sirene, klakson mobil, gonggongan anjing, atau suara tembakan. 
Setelah suara terdeteksi dan diklasifikasikan, sistem dapat memberikan notifikasi dalam bentuk visual atau getaran kepada pengguna Tuna Rungu, sehingga mereka dapat segera menyadari ancaman dan mengambil tindakan antisipasi. 
Dengan demikian, teknologi ESC berbasis \textit{Deep Learning} memiliki potensi besar untuk meningkatkan keselamatan dan kualitas hidup penyandang Tuna Rungu dalam mobilitas sehari-hari di lingkungan perkotaan. \par

% PARAGRAF 4: SOLUSI TEKNIS (PANNs & VARIABEL)
Namun, pengembangan model ESC yang mampu mendeteksi bahaya secara akurat menghadapi tantangan teknis, terutama terkait kebutuhan dataset berskala besar. 
Oleh karena itu, metode \textit{Transfer Learning} menjadi solusi efektif untuk melatih model deteksi, dengan memanfaatkan kerangka kerja dan bobot model dari \textit{Pre-trained Audio Neural Networks} (PANNs). 
PANNs dipilih karena tidak hanya menyediakan bobot model dari dataset masif \cite{kong2020panns}, tetapi juga menyediakan variasi arsitektur yang dirancang khusus untuk menangani berbagai representasi input audio. 
Hal ini memunculkan urgensi untuk mengevaluasi arsitektur mana yang paling optimal, apakah arsitektur yang bekerja pada domain waktu dengan input \textit{Raw Waveform} (satu dimensi), atau arsitektur pada domain frekuensi dengan input \textit{Spectrogram} (dua dimensi). 
Selain itu, PANNs juga menyediakan arsitektur dengan pendekatan \textit{Hybrid} yang menggabungkan kedua representasi tersebut, yang secara teoritis berpotensi memaksimalkan akurasi deteksi. \par

% PARAGRAF 5: RESEARCH GAP (ARGUMEN ILMIAH UTAMA) -> INI YANG DIPECAH
Meskipun Kong et al. telah memaparkan tolak ukur kinerja model-model tersebut pada dataset masif AudioSet \cite{kong2020panns}, performa tersebut belum tentu linier ketika diterapkan pada kasus penerapan spesifik dengan ketersediaan data yang terbatas (\textit{data scarcity}) seperti pada kasus klasifikasi suara lingkungan perkotaan. 
Kompleksitas arsitektur model \textit{Hybrid} dan \textit{Spectrogram}, misalnya, memiliki risiko \textit{overfitting} yang lebih tinggi dibandingkan model \textit{Raw Waveform} ketika dilatih pada dataset yang kecil. 
Oleh karena itu, penelitian ini menjadi krusial untuk mengevaluasi ulang adaptabilitas dan generalisasi ketiga arsitektur tersebut secara spesifik pada dataset UrbanSound8K. \par

% PARAGRAF 6: TUJUAN PENELITIAN
Untuk menjawab tantangan adaptabilitas pada dataset terbatas tersebut, penelitian ini bertujuan utama secara teknis untuk menginvestigasi dan membandingkan kinerja tiga pendekatan representasi input, yaitu \textit{Raw Waveform}, \textit{Spectrogram}, dan \textit{Hybrid}. 
Studi komparasi ini diposisikan sebagai langkah fundamental untuk menemukan konfigurasi model yang paling \textit{robust} (tahan uji) terhadap minimnya data, sekaligus meminimalisir kesalahan deteksi fatal. 
Dengan demikian, hasil evaluasi ini diharapkan dapat menjadi landasan teknis yang valid bagi pengembangan teknologi asistif yang benar-benar andal untuk menjamin keselamatan komunitas Tuna Rungu.

\section{Rumusan Masalah} \label{I.Rumusan Masalah}
Berdasarkan latar belakang yang telah diuraikan, maka rumusan masalah dalam penelitian ini adalah: \par

\begin{enumerate}[noitemsep]
  \item Bagaimana pengaruh perbedaan representasi input (\textit{Raw Waveform}, \textit{Spectrogram}, dan \textit{Hybrid}) terhadap performa model \textit{Pre-trained Audio Neural Networks} (PANNs) dalam mengklasifikasikan suara bahaya pada kondisi ketersediaan data yang terbatas?
  \item Representasi input manakah yang menghasilkan model paling optimal (berdasarkan metrik \textit{Accuracy}, \textit{Precision}, \textit{Recall}, dan \textit{F1-Score}) untuk meminimalisir kesalahan deteksi pada sistem keselamatan penyandang Tuna Rungu?
\end{enumerate}

\section{Tujuan Penelitian} \label{I.Tujuan}
Tujuan dari penelitian ini adalah : \par

\begin{enumerate}[noitemsep]
  \item Menganalisis pengaruh perbedaan representasi input (\textit{Raw Waveform}, \textit{Spectrogram}, dan \textit{Hybrid}) terhadap performa model \textit{Pre-trained Audio Neural Networks} (PANNs) dalam mengklasifikasikan suara bahaya pada kondisi ketersediaan data yang terbatas.
  \item Menentukan pendekatan representasi input yang menghasilkan model paling optimal (berdasarkan metrik \textit{Accuracy}, \textit{Precision}, \textit{Recall}, dan \textit{F1-Score}) untuk meminimalisir kesalahan deteksi pada sistem keselamatan penyandang Tuna Rungu.
\end{enumerate}

\section{Batasan Masalah} \label{I.Batasan}
Batasan masalah yang didefinisikan dalam penelitian ini adalah sebagai berikut : \par

\begin{enumerate}[noitemsep]
  \item Penelitian ini menggunakan dataset publik \textit{UrbanSound8K} sebagai representasi suara lingkungan perkotaan. Penulis tidak melakukan pengambilan data primer (perekaman langsung di lapangan).
  \item Fokus klasifikasi dibatasi pada kelas suara yang berpotensi menjadi ancaman atau peringatan bagi keselamatan fisik di jalan raya, yang diambil dari dataset \textit{UrbanSound8K} (\textit{car\_horn}, \textit{siren}, \textit{dog\_bark}, \textit{gun\_shot}, dan \textit{drilling}).
  \item Penelitian ini berfokus pada eksperimen pelatihan (\textit{training}) dan evaluasi performa model \textit{Deep Learning}. Penelitian ini \textbf{tidak mencakup} pembuatan perangkat keras (\textit{hardware}), pengembangan aplikasi antarmuka pengguna (\textit{User Interface}) berbasis \textit{mobile/web}, maupun implementasi sistem secara \textit{real-time}.
  \item Kerangka kerja yang digunakan adalah \textit{Pre-trained Audio Neural Networks} (PANNs) dengan arsitektur spesifik: \textit{Res1dNet31} (untuk \textit{Raw Waveform}), \textit{ResNet38} (untuk \textit{Spectrogram}), dan \textit{Wavegram\_Logmel\_CNN14} (untuk \textit{Hybrid}).
  \item Analisis kinerja model dilakukan menggunakan \textit{Confusion Matrix}, dengan parameter pengukuran kuantitatif meliputi \textit{Accuracy}, \textit{Precision}, \textit{Recall}, dan \textit{F1-Score}.
\end{enumerate}

\section{Manfaat Penelitian} \label{I.Manfaat}
Manfaat dari penelitian ini adalah : \par

\begin{enumerate}[noitemsep]
	\item Memberikan bukti nyata terkait efektivitas metode \textit{Transfer Learning} pada arsitektur PANNs serta perbandingan performa antara representasi input \textit{Raw Waveform}, \textit{Spectrogram}, dan \textit{Hybrid} dalam mengatasi keterbatasan dataset.
	\item Berkontribusi dalam pengembangan teknologi asistif berbasis AI yang dapat meningkatkan keselamatan dan kemandirian mobilitas penyandang Tuna Rungu melalui deteksi suara bahaya yang akurat.
	\item Menjadi referensi bagi penelitian selanjutnya atau pengembang aplikasi dalam menentukan konfigurasi model yang paling optimal untuk diterapkan pada sistem peringatan dini.
\end{enumerate}


\section{Sistematika Penulisan} \label{I.Sistematika}
Sistematika penulisan berisi pembahasan apa yang akan ditulis disetiap Bab. Sistematika pada umumnya berupa paragraf yang setiap paragraf mencerminkan bahasan setiap Bab. \par

\subsection*{Bab I}
Bab ini berisikan penjelasan latar belakang dari topik penelitian yang berlangsung, rumusan masalah dari masalah yang dihadapi pada penjelasan di latar belakang, tujuan dari penelitian, batasan dari penelitian, manfaat dari hasil penelitian, dan sistematika penulisan tugas akhir. \par

\subsection*{Bab II}
Bab ini membahas mengenai tinjauan pustaka dari penelitan terdahulu dan dasar teori yang berkaitan dengan penelitian ini.

\subsection*{Bab III}
Bab ini berisikan penjelasan alur kerja sistem, alat dan data yang digunakan, metode yang digunakan, dan rancangan pengujian.

\subsection*{Bab IV}
Bab ini membahas hasil implementasi dan pengujian dari penelitian yang dilakukan, serta analisis dan evaluasi yang dapat dipetik dari hasil.

\subsection*{Bab V}
Bab ini membahas kesimpulan dari hasil penelitian dan juga saran untuk penelitian selanjutnya.